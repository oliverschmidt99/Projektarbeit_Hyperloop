\chapter{Implementierung}

\section{Distanzmessung}

Der Sensor von Pepperl+Fuchs wurde ursprünglich für die Distanzmessung angeschafft. Da dieser Sensor jedoch sehr teuer ist und wir nicht das Risiko eines möglichen Schadens im Vakuum eingehen wollten, wurde nach einer kostengünstigen Alternative gesucht.\\
Die Idee bestand darin, ein Entfernungsmessgerät von PARKSIDE zu verwenden und den Sensor aus dem Gerät zu entfernen. Die Messwerte werden anschließend mit einer MCU decodiert.

\subsection{Verbindung}
Der Sensor ist an der Hauptplatine mit einem Flachbandkabel angeschlossen wie in der Abbildung \ref{img_2_2:sen_dis_parkside:1} zu sehen ist. An der Hauptplatine sind 4 Lötlöcher die Frei sind 
Mit dem Ozilloskop haben wir die Einzelnen Leitungen gemessen, dabei waren die.


\begin{figure}[ht]
	\begin{center}
		\includegraphics[width=1\textwidth]{img/2_sen/dis_parkside_1_outside.png}
		\caption{PARKSIDE – Distanzsensor – Innenaufbau}
		\label{img_2_2:sen_dis_parkside:1}
	\end{center}
\end{figure}


\begin{table}[ht]
	\centering
	\caption{PARKSIDE – Pin Mapping – Distanzsensor}
	\label{parkside:pinmapping}
	\begin{tabular}{l|ll}
		\hline
		\textbf{Pin} & \textbf{Farbe} & \textbf{Funktion} \\ \hline
		1            & Rot            & 3V3               \\
		2            & Weiß           & RX (receiver)     \\
		3            & Gelb           & TX (transmitter)  \\
		4            & Schwarz        & GND               \\ \hline
	\end{tabular}
\end{table}




\begin{itemize}
	\item Anschlüsse herauszufinden vier anschlüsse gefunden.
	\item Was für Aufgaben haben die Anschlüsse?
	      \subitem Was für ein Kommunikationsprotokol hat der Sensor?
	      \subitem Asynchron und seriell mit Baudrate 115200. (Ozilloskop)
	\item mit einem ESP32 die Sensordaten lesen.
	\item Nachrichtdekodierung (Controller MCU)
	      \subitem Ox24 Start (Nachricht start)
	      \subitem 0x26 Stopp (Nachricht ende)
	      \subitem 24 30 30 30 33 32 36 30 30 32 39 26 (Stopp signal)
	      \subitem 24 30 30 30 33 32 36 30 31 33 30 26 (Laser an)
	      \subitem 24 30 30 30 32 32 31 32 33 26 (Messen)
\end{itemize}


