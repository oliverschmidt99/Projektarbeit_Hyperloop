\chapter{Konklusion}


In dieser Projektarbeit wurde erfolgreich ein funktionsfähiges Fahrzeug für den Hyperloop-Prototyp entwickelt und dessen Aufbau sowie die Steuerung detailliert beschrieben. Die Implementierung umfasste mehrere wesentliche Schritte: Von der Erstellung eines Schaltplans für die elektrische Verdrahtung über die Integration der Sensorik bis hin zur Programmierung und Simulation der Steuerungslogik.

Die Implementierung der Steuerung erfolgte über ein Speedgoat-System, das Echtzeitsteuerung ermöglicht. Die Simulation der Steuerungslogik in Simulink stellte sicher, dass das Fahrzeug sowohl im manuellen als auch im automatischen Modus betrieben werden kann. Im manuellen Modus kann der Benutzer das Fahrzeug vorwärts und rückwärts fahren lassen, während der automatische Modus das Fahrzeug beschleunigt und bei Annäherung an das Streckenende abbremst. Zudem wurde die Distanzmessung durch einen alternativen Sensor kostengünstig und zuverlässig realisiert, was die Positionierung des Fahrzeugs in der Röhre ermöglicht.

Die Wahl der Bauteile, wie des BLDC-Motors und der IO-Module, sowie die Verdrahtungskonventionen wurden sorgfältig dokumentiert und implementiert, um eine sichere und stabile Betriebsumgebung zu gewährleisten. Die entwickelten Endschnittstellen bieten eine Grundlage für zukünftige Erweiterungen und Testdurchläufe.

Zukünftige Arbeiten könnten darauf abzielen, das Fahrzeug unter realistischen Bedingungen zu testen, um das Zusammenspiel von Steuerung und Sensorik weiter zu optimieren. Die durchgeführten Implementierungen und die offene Architektur des Systems bieten somit eine flexible Basis für die Weiterentwicklung des Hyperloop-Konzepts und die Erforschung einer innovativen, energieeffizienten Transportlösung.